\documentclass{article}

\usepackage[T2A]{fontenc}
\usepackage[serbianc]{babel}
\usepackage{hyperref}
\usepackage{algpseudocode}
\usepackage{amsmath}
\usepackage{graphicx}
\usepackage{multicol}
\usepackage{pgfplots}
\usepackage{float}

\pgfplotsset{compat = newest}

\begin{document}

% naslovna strana

\thispagestyle{empty}

\graphicspath{{./images/}}

\begin{center}
    \includegraphics[scale=0.3]{ZIG}
\end{center}

\vspace*{20mm}

\begin{center}
    \Large Матурски рад из Програмирања

    \vspace*{8pt}

    \Huge \textbf{Хеуристичко програмирање}
\end{center}

\vspace*{40mm}

% TODO: namestiti font size
\setlength{\columnsep}{50pt}
\begin{multicols}{2}
 {\noindent \large Ученик:
\\Иван Бокор  IV$_{12}$}

 {\noindent \hfill \large \centering{}Ментор: \\
 \hfill \phantom{111} Вукман Кораћ}

\end{multicols}

\vfill

\begin{center}
    \Large Београд, мај 2023.
\end{center}

\newpage

% sadrzaj

\tableofcontents

% rad

\newpage
\pagenumbering{arabic}

\section{Увод}

Aлгоритам представља низ дефинисаних корака (или операција) који воде ка
решавању неког проблема. Алгоритмима су се дуго бавили математичари, али је 
тек након конструисања првих рачунара њима посвењена значајна пажња. Током 
20. века осмишљени су алгоритми за решавање разних сложених проблема.  Међутим,
неки проблеми су били сувише комплексни те су алгоритми осмишњени за њихово
решавање били неефикасни - захтевали су пуно меморијског простораили је просто 
било потребно пуно времена за њихово извршавање (нпр. Проблем трговачког путника).
Стога су осмишљени алгоритми који повећавају ефикаснот на рачун тачности решења.
Такви алгоритми се често називају хеуристике а коришћење тих метода се зове хеуристичко програмирање.
У овом раду ћемо се бавити генетским алгоритмима који припадају тој категорији.

Генетски алгоритми (у даљем тексту ГА) представљају хеуристичку методу оптимизације[1] који моделују
механизме еволуције врста из природе. Идеја је да почнемо од неког скупа насумичних решења и онда
еволуцијом добијемо оптимално решење. Еволуција се одвија помоћу механизма \textbf{укрштања}
и \textbf{мутације}. Укрштањем два решења добијамо ново потенцијално решење проблема а мутацијом
подижемо степен различитости у целом скупу решења. Процес бирања решења која ћемо укрстити се назива 
\textbf{селекција}. Пошто је терминологија готово иста као она из биологије, скуп потенцијалних 
решења неког проблема ћемо називати \textbf{популацијом} а неко конкретно решење \textbf{јединком}. 
Сваку јединка је одређена конкдретним \textbf{генетским кодм} (такође се користе термини ген, хромозом) који енкодира 
решење проблема. С обзиром да  ГА моделује еволуцију, има смисла говорити и о генерацијама. Генерацију 
представљају све јединке које су се нашле у датом тренутку у популацији. У почетној генерацији се 
налазе оне јединке које су насумично генерисане, а свака следећа настаје применом еволутивних 
механизама (укрштања и мутације) на ону претходну. 

Математички говорећи, ГА је оптимизациона метода усмереног случајног претраживања
простора решења. Циљ алгоритма је пронаћи глобално решење неког проблема, што се 
може поистоветити са тражењем глобалног минимума или максимума неке функције. Ако
осмислимо неку функцију која ће евалуирати потенцијална решења проблема до којих 
генетски алгоритам буде дошао, оптимално решење ћемо управо наћи у максимуму или
минимуму (у зависности од типа проблема који се решава). Зато дефинишемо \textbf{функцију доброте}
$fitness(x)$ или \textbf{функцију грешке} $cost(x)$ која на неки начин евалуира наша решења, 
тј. сваком решењу придружује неки реалан број којим меримо колико је дато решење "добро"
(отуда и назив функција доброте). Како би смо били сигурни да је алгоритам стигао до
оптималног решења, унапред ћемо одредити колико ће итерација алгоритам извршити. За то
дефинишемо глобални парамтер који чува тај број итерација. Осим броја итерација, глобалним 
параметрима су одређју: величина популације, проценат популације који мутира, величина и дужина хромозома,
... У одељку 2.1 ће бити испитано како ови параметри утичу на перформансе ГА.

У овом раду биће приказан начин рада генетског алгоритма решавањем два проблема: први проблем
је тражење жељене ниске а други тражење максимума неке функције. Први проблем служи за 
упознавање са функционисањем генетског алгоритма, док други служи за његово тестирање и 
евалуацију. За оба проблема испрограмиран је генетски алгоритам у програмском језику 
C++. Код за оба алгоритма је могуће пронаћи у GitHub репозиторијуму на следећем
линку: \url{github.com/blin04/maturski}.

\section{Пример рада генетског алгоритма}

Начин рада генетског алгоритма ћемо објаснити на проблему претраге ниски. Односно,
алгоритам ће имати задатак да "пронађе" неку задату ниску, нпр. \textit{`Земунска гимназија`}.
Ниску коју треба наћи се уноси приликом покретања програма и чува се у глобалном
параметру $GOAL$. Дужина те ниске се означава са $n$.

\subsection{Иницијализовање популације}
За почетак, потребно је иницијализовати скуп, тј. популацију, потенцијалних решења.
Пошто може бити унета било каква ниска, почетни скуп садржи насумично генерисане ниске.
У популацији се налази 200 јединки. Алгоритам ћемо поновити 500 пута како би смо били сигурни
да је наше решење оно најоптималније. Следеће шта треба одредити јесте како ће   

\subsection{Репрезентација јединки}
Као што су људи једнозначно одређени ДНК молекулима, тако су и јединке у популацији
одређеne неким генетским кодом. Генетски код може да буде било шта, под условом да
је могуће на основу њега конструисати потенцијално решење проблема. Понекад генетски код
чини управо то потенцијално решење. Обично се користе бројеви, ниске или неке друге 
структре са којима рачунари лако манипулишу. У овом слуају, генетски код јединке
ће бити управо та ниска коју она `представља`. То значи да дужина генетског кода мора бити $n$.

\subsection{Селекција}
Процес селекције служи да би се изабрали родтиељи - јединке од којих ће настати нове
2 јединке (тј. 2 нова решења). Како би алгоритам стигао до оптималног решења, потребно 
је да с бирају што бољи родитељи. За то служе функција добороте или функција грешке које вреднују јединке.
У овом случају користићемо функцију грешке, која евалуира колико је нека јединка лоша. 
Функција је дефинисана на следећи начин: 

\begin{equation*}
    cost(s)=\sum_{i=1}^{n} |s_i - GOAL_i| 
\end{equation*}

где је $s$ потенцијално решење а $GOAL$ ниска коју желимо достићи. Овиме постижемо да 
јединке које представљају ниске чији су карактери ближи карактерима (по ASCII вредности) коначне ниске имају 
мању вредност функцију грешке (напоменимо само да оваква дефиниција функције грешке није 
баш сасвим добра, али ће послужити за демонстрацију генетског алгоритма). Пошто
је циљ да се бирају што бољи родитељи, сваком родитељу се додељује нека вероватноћа да буде изабран
која ће бити обрнуто пропорционална функцији грешке ($p_i \propto \frac{1}{cost(x)}$).
Избор родитеља се онда врши Roulette wheel алгоритмом. Тај алгоритам се може објаснити као
као да имамо неки точак са показивачем на врху, који је издељен на онолико делова колико има јединки у популацији
(делови изгледају као парћићи пице). Површина сваког дела је пропорционална вероватноћи да 
та јединка буде изабрана за родтиеља. Бирање родитеља се врши тако што се точак заврти, па 
се изабере она јединка на коју показује показивач након заустављања круга. Овакав начин селекције
се зове једноставна селекција и захтева да приликом укрштања направимо нови низ у који убацујемо
новодобијене јединке. Прављење новог низа се може избећи коришћењем елиминацијске селекције, о чему ће бити
речи у одељку 3.2.
\newline

\begin{algorithmic}
\State $r \gets random(0$, $TOTAL\_COST)$
\State $suma \gets 0$
\State $izabran \gets NULL$
\For{$i \gets 1...n$}
    \State $suma \gets suma + cost(p_i)$
    \If{$r \geq suma$}
        \State $izabran \gets p_i$
    \EndIf
\EndFor

\end{algorithmic}


\subsection{Укрштање}
Након што се родитељи изаберу, потребно је извршити укрштање генетских кодова родтиеља
како би се добио нови генетски код који представља нову јединку. Постоје разни алгоритми укрштања
и обично у њима постоји одређена доза насумичности, како би се симулирао процес укрштања гена у природи.
За потребе ове демонстрације имплементиран је алгоритам који функционише на следећи начин: 
за неку позицију $i$ $(i \in [1, n])$ пореде се карактери који се налазе на том месту у генетском коду родтиеља
и у генетски код детета се додаје онај који је ближи одговарајућем карактеру у коначној ниски. 
\newpage
% mozda obrisati?
\begin{algorithmic}

\State $n \gets len(GOAL)$

\For{$i \gets 1...n$ }
    \If{$|p1_i - GOAL_i| \leq |p2_i - GOAL_i|$}
        \State $ch \gets ch + p1_i$
    \Else
        \State $ch \gets ch + p2_i$
    \EndIf
\EndFor
\end{algorithmic}

\subsection{Мутација}
Мутација има изузетно значајну улогу у раду генетског алгоритма. Пошто је у уводу речено
да оно што ГА ради јесте у суштини тражење минимума (или максимума) неке функције, јавља се 
проблем да алгоритам заврши у неком локалном минимуму (максимуму) и не пронађе 
оптимално решење. Такве ситуације се настоје спречити мутацијом. Мутација, као и у природи,
подразумева насумичну промену генетског кода. Пошто се она у природи ретко дешава, обично
се ГА праве тако да мутира $\leq 5 \%$ популације. Насумична промена генетског кода 
може довести до тога да се јединка помери од локалне екстремне тачке чиме се повећава 
шанса за проналаском оне глобалне.

\subsection{Добијена решења}
Алгоритму је дато да пронађе ниску `\textit{Земунска Гимназија}`. Решења алгоритма су варирала
због насумичне иницијализације почетне популације, али су се минимално разликовала од тражене ниске.
Ове минималне разлике је могуће уклонити бољом дефиницијом функције грешке као и бољим алгоритмом укрштања.
У слици Х. приказан је поступак тражења.


\section{Тестирање генетског алгортима}
У овом делу рада биће укратко објашњен начин рада генетског алгоритма који проналази
максимум неке задате функције (поступак проналаска минимума је еквивалентан поступку
проналаска максимума). Формално, треба пронаћи реалан број $x$ за који важи $x \in [a, b]$ и 
$f(x) = max$ за неку задату функцију $f(x)$ која је дефинисана на интервалу $[a, b]$. 
У одељку 4. биће приказано како перформансе овог алгоритма зависе од различитих 
вредности глобалних параметара.

\subsection{Грејево кодирање и иницијализација популације}
Избор генетског кода је у овом случају мало тежи. Уколико би смо покушали да као генетски код
просто користимо онај реални број који јединка представља, било би поприлично тешко 
имплементирати укрштање и мутацију. Испоставља се да се ово може заобићи ако користимо
бинарне бројеве. Бинарни бројеви записују се помоћу цифара 0 и 1 и врло су погодни за 
наведене операције због битовских операција (and, or, not, shift). Ако за генетски код
искористимо бинарни број, лако конструишемо одговарајући реалан број који је њиме представљен:

\begin{equation} \label{eq1}
    x = a + \frac{t}{2^n - 1} (b - a)
\end{equation}

где је $t$ генетски код а $n$ његова дужина. Једини проблем који се јавља са оваким
генетским кодом је следећи: нека имамо у популацији јединку са генетским кодом 01111 а нека се максимум
налази у тачки која је представљена генетским кодом 10000. Дакле, у популацији постоји
јединка која је јако близу коначном решењу, али због начина репрезентације бинарних 
бројева потребно је променити чак 4 бита. Ово је проблем јер бројеви који су 
јако близу (готово један поред другог) имају доста различите репрезентације. Као решење
можемо искористити Грејево кодирање, јер се суседни бројеви представљени овим кодирањем 
разликују у само једном биту. Бинарни број $t = t_1 t_2 \dots t_n$ можемо превести у 
Грејев код $g = g_1 g_2 \dots g_n$ на следећи начин: 

\begin{align*}
    g_k &= b_k \oplus b_{k + 1} & k &= 0, 1, \dots, n - 1 \\
    g_n &= b_n                   &   k &= n
\end{align*}

Дакле, генетски код ће бити бинарни бројеви који су записани Грејевим кодирањем. 
Одговарајуће реални број који је представљен неким генетским кодом добијамо тако 
што тај код у бинарни број, а онда убацимо у једначину \ref*{eq1}.

\subsection{Селекција, укрштање и мутација}
Селекција је извршена на два начина. Први начин је једноставна селекција je
објашњена у одељку 2.3. У овом случају користимо функцију доброте и то ће бити управо 
та функцијa чији максимум тражимо плус нека константа. Та константа нам служи како не 
би смо имали негативне вредности функције доброте (то је услов за алгоритам селекције). 
Ако је функција која нам је задата позитивна на интервалу $[a, b]$ онда је вредност 
ове константе 0. У супротном, она добија следећу вредност $CONST = |\min(f(x_1), f(x_2), ..., f(x_n))|$.
Други начин је елиминацијска селекција. Она се од једноставне разликује у томе што се пров
обрише $М$ најгорих јединки из популације, а након тога се укрштањем оних преосталих додаје
$М$ нових јединки. Избор јединки за елиминацију се такође врши Roulette wheel алгоритмом.
Функција доброте остаје иста, само што сада нема потребе уводити константу која осигурава да је вредност функције позитивна.

Укрштање два генетска кода се врши бит по бит, сваки бит има вероватноћу  $p$
да добије вредност од провг родитеља а $1 - p$ да добије вредност другог. У овом
случају је коришћено $p = 0.5$. Мутација је поново одређена глобалним парамтером $PM$ 
и функционише тако што се насумично промени један бит у хромозому. Коришћено је 
$PM = 3\%$.

% mozda dodati pseudokod?

\subsection{Приказ рада алгоритма}
Како бисмо приказали рад алгоритма, приказаћемо претрагу графички. Графици су добијени на основу података
који су се нашли у популацији у неким генерацијама. Коришћена је величина популације 20 како графици не би били
претрпани.

\begin{figure}[H]
    \centering
    \begin{tikzpicture}

    \begin{axis} [
        xmin = -2.1, xmax = 2.1,
        ymin = -5.0, ymax = 5.0,
        grid = both,
        major grid style = {lightgray},
        minor grid style = {lightgray!25}, 
        ]
        % funkcija
        \addplot [
            domain = -3.0:3.0,
            samples = 200,
            smooth,
            thick,
            blue,
        ] {x^5 - 5*x^3 - (1/3) * x^2 + 4 * x + 1};
        %podaci
        \addplot[red, only marks] table {../source/function_maximum/results/gen00};
    \end{axis}

    \end{tikzpicture}

    \caption{Иницијализована популација}
\end{figure}

\begin{figure}[H]
    \centering
    \begin{tikzpicture}

    \begin{axis} [
        xmin = -2.1, xmax = 2.1,
        ymin = -5.0, ymax = 5.0,
        grid = both,
        major grid style = {lightgray},
        minor grid style = {lightgray!25}, 
        ]
        % funkcija
        \addplot [
            domain = -3.0:3.0,
            samples = 200,
            smooth,
            thick,
            blue,
        ] {x^5 - 5*x^3 - (1/3) * x^2 + 4 * x + 1};
        %podaci
        \addplot[red, only marks] table {../source/function_maximum/results/gen02};
    \end{axis}

    \end{tikzpicture}

    \caption{Друга Генерација}
\end{figure}

\begin{figure}[H]
    \centering
    \begin{tikzpicture}

    \begin{axis} [
        xmin = -2.1, xmax = 2.1,
        ymin = -5.0, ymax = 5.0,
        grid = both,
        major grid style = {lightgray},
        minor grid style = {lightgray!25}, 
        ]
        % funkcija
        \addplot [
            domain = -3.0:3.0,
            samples = 200,
            smooth,
            thick,
            blue,
        ] {x^5 - 5*x^3 - (1/3) * x^2 + 4 * x + 1};
        %podaci
        \addplot[red, only marks] table {../source/function_maximum/results/gen08};
    \end{axis}

    \end{tikzpicture}

    \caption{Осма Генерација}
\end{figure}

\begin{figure}[H]
    \centering
    \begin{tikzpicture}

    \begin{axis} [
        xmin = -2.1, xmax = 2.1,
        ymin = -5.0, ymax = 5.0,
        grid = both,
        major grid style = {lightgray},
        minor grid style = {lightgray!25}, 
        ]
        % funkcija
        \addplot [
            domain = -3.0:3.0,
            samples = 200,
            smooth,
            thick,
            blue,
        ] {x^5 - 5*x^3 - (1/3) * x^2 + 4 * x + 1};
        %podaci
        \addplot[red, only marks] table {../source/function_maximum/results/gen60};
    \end{axis}

    \end{tikzpicture}

    \caption{Шездесета Генерација}
\end{figure}

\section{Резултати}
Zavisnosti brzine i tacnosti od parametara
Tabela

\section{Закључак}

\section{Литература}

\end{document}